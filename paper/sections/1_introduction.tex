\section{Introduction}\label{sec:introduction}
In the last few years, cloud computing~\cite{cloud_cite} has become a popular solution for running data-intensive
applications remotely. However, large scale data-centres are not a feasible in application domains that require low
latency or high security and privacy. To deal with such domains, \emph{fog/edge computing} has emerged as a
complementary paradigm allowing tasks to be executed at the edge of networks, close to the user, in small data-centers,
known as \emph{edge clouds}.

As the Internet-of-things grows, fog/edge cloud computing is a key enabling technology in particular for applications
in smart cities, disaster response scenarios, home automation systems, etc. In these applications, low-powered devices
generate data or tasks that can't be processed locally but are impractical to use with standard cloud computing
services. More specifically, in smart cities, these devices could be smart traffic light systems that collect data from
roadside sensors to optimise the traffic light sequence to minimise vehicle waiting times; or to analyse videos feeds
from CCTV cameras of suspicious behaviour. In disaster response, sensor data from autonomous vehicles can be aggregated
to produce real-time maps of devastated areas to help first responders better understand the situation and search for
survivors.

To accomplish these tasks, there are typically several types of resources that are needed, including communication
bandwidth, computational power and data storage resources~\cite{vaji_infocom}, and tasks are generally
delay-sensitive, i.e., have a specific completion deadline. When accomplished, different tasks carry different values
for their owners (e.g., the users of IoT devices or other stakeholders such as the police or traffic authority). This
value will depend on the importance of the task, e.g., analysing current levels of air pollution may be less important
than preventing a large-scale traffic jam at peak times or tracking a criminal on the run. Given that edge clouds are
often highly constrained in their resources~\cite{edge_limitations}, we are interested in allocating tasks to edge
cloud servers to maximize the overall social welfare achieved (i.e., the sum of completed task values). This is
particularly challenging, because users in edge clouds are typically self-interested and may behave
strategically~\cite{Bi2019} or may prefer not to reveal private information about their values to a central allocation
mechanism~\cite{Pai2013}.

An important shortcoming of existing work of resource allocation in edge cloud computing is that it assumes tasks have
strict resource requirements -- that is, each task must be allocate a fixed amount of CPU cycles or bandwidth by a
server. This can cause bottlenecks for certain resources when multiple tasks over-request resources. However, in
practice tasks have flexibility in how resources are allocated given the task is computed within a time limit. This
ability to flexibility allocate resource is additionally important in the case of edge computing due to the limited
resource capacities that servers have in comparison to large data-centre. Using this idea of task resource flexibility,
in this paper, we proposed a new optimisation problem that is incompatible with previous research. Therefore three
mechanisms are proposed for the problem: a greedy mechanism, an incentive compatible centralised auction and a novel
decentralised iterative auction that achieve ~95\% of the theoretical optimal.

%% TODO Make an addition to why this flexibility breaks everything previously


\section{Related work}\label{sec:related-work}
There is a considerable amount of research in the area of resource allocation and pricing in cloud computing, some of
which use auction mechanisms to deal with competition~\cite{KUMAR2017234,Zhang2017,Du2019,Bi2019}.
However, these approaches assume that users request a fixed amount of resources system resources and processing rates,
with the cloud provider having no control over the speeds, only the servers that the task was allocated to. In our
work, tasks' owners report deadlines and overall data and computation requirements, allowing the edge cloud server to
distribute its resources more efficiently based on each task's requirements.

Other closely related work on resource allocation in edge clouds~\cite{vaji_infocom} considers both the placement of
code/data needed to run a specific task, as well as the scheduling of tasks to different edge clouds. The goal there is
to maximize the expected rate of successfully accomplished tasks over time. Our work is different both in the setup and
the objective function. Our objective is to maximize the value over all tasks. In terms of the setup, they assume that
data/code can be shared and they do not consider the elasticity of resources.

Our problem is related to multidimensional knapsack problems. In particular~\cite{Nip2017}, consider flexibility in
the allocation, with linear constraints that are used for elastic weights. The paper provides a pseudo-polynomial time
complexity algorithm for solving this problem to maximize the values in the knapsack. Our optimisation problem case is
similar to their, but differ due to constraints with our using non-linear not linear constraints.
% Link to share - https://www.overleaf.com/3134569646ccdhjxcyjzyr

%%
%% This is file `sample-manuscript.tex',
%% generated with the docstrip utility.
%%
%% The original source files were:
%%
%% samples.dtx  (with options: `manuscript')
%%
%% IMPORTANT NOTICE:
%%
%% For the copyright see the source file.
%%
%% Any modified versions of this file must be renamed
%% with new filenames distinct from sample-manuscript.tex.
%%
%% For distribution of the original source see the terms
%% for copying and modification in the file samples.dtx.
%%
%% This generated file may be distributed as long as the
%% original source files, as listed above, are part of the
%% same distribution. (The sources need not necessarily be
%% in the same archive or directory.)
%%
%% The first command in your LaTeX source must be the \documentclass command.
%%%% Small single column format, used for CIE, CSUR, DTRAP, JACM, JDIQ, JEA, JERIC, JETC, PACMCGIT, TAAS, TACCESS, TACO, TALG, TALLIP (formerly TALIP), TCPS, TDSCI, TEAC, TECS, TELO, THRI, TIIS, TIOT, TISSEC, TIST, TKDD, TMIS, TOCE, TOCHI, TOCL, TOCS, TOCT, TODAES, TODS, TOIS, TOIT, TOMACS, TOMM (formerly TOMCCAP), TOMPECS, TOMS, TOPC, TOPLAS, TOPS, TOS, TOSEM, TOSN, TQC, TRETS, TSAS, TSC, TSLP, TWEB.
% \documentclass[acmsmall]{acmart}

%%%% Large single column format, used for IMWUT, JOCCH, PACMPL, POMACS, TAP, PACMHCI
% \documentclass[acmlarge,screen]{acmart}

%%%% Large double column format, used for TOG
% \documentclass[acmtog, authorversion]{acmart}

%%%% Generic manuscript mode, required for submission
%%%% and peer review
\documentclass[manuscript,screen]{acmart}

%%
%% \BibTeX command to typeset BibTeX logo in the docs
\AtBeginDocument{%
\providecommand\BibTeX{{%
\normalfont B\kern-0.5em{\scshape i\kern-0.25em b}\kern-0.8em\TeX}}}

%% Rights management information.  This information is sent to you
%% when you complete the rights form.  These commands have SAMPLE
%% values in them; it is your responsibility as an author to replace
%% the commands and values with those provided to you when you
%% complete the rights form.
%% TODO
%\setcopyright{acmcopyright}
%\copyrightyear{2018}
%\acmYear{2018}
%\acmDOI{10.1145/1122445.1122456}

%% These commands are for a PROCEEDINGS abstract or paper.
%% TODO
%\acmConference[Woodstock '18]{Woodstock '18: ACM Symposium on Neural
%Gaze Detection}{June 03--05, 2018}{Woodstock, NY}
%\acmBooktitle{Woodstock '18: ACM Symposium on Neural Gaze Detection,
%June 03--05, 2018, Woodstock, NY}
%\acmPrice{15.00}
%\acmISBN{978-1-4503-XXXX-X/18/06}

\usepackage{listings}
\usepackage{float}
\usepackage{algorithm}
\usepackage{algorithmic}
\usepackage{appendix}
\usepackage{soul}
\usepackage[]{todonotes}
\usepackage{amsmath}
\usepackage{wrapfig}

%%
%% Submission ID.
%% Use this when submitting an article to a sponsored event. You'll
%% receive a unique submission ID from the organizers
%% of the event, and this ID should be used as the parameter to this command.
%%\acmSubmissionID{123-A56-BU3}

%%
%% The majority of ACM publications use numbered citations and
%% references.  The command \citestyle{authoryear} switches to the
%% "author year" style.
%%
%% If you are preparing content for an event
%% sponsored by ACM SIGGRAPH, you must use the "author year" style of
%% citations and references.
%% Uncommenting
%% the next command will enable that style.
%%\citestyle{acmauthoryear}

%%
%% end of the preamble, start of the body of the document source.
\begin{document}

    %% The "title" command has an optional parameter,
    %% allowing the author to define a "short title" to be used in page headers.
    \title{Auction-based Mechanisms for Resource-elastic Tasks in Edge Cloud Computing}

    %% The "author" command and its associated commands are used to define
    %% the authors and their affiliations.
    %% Of note is the shared affiliation of the first two authors, and the
    %% "authornote" and "authornotemark" commands
    %% used to denote shared contribution to the research.
    \author{Mark Towers}
    \email{mt5g17@soton.ac.uk}
    \affiliation{
    \institution{University of Southampton}
    \city{Southampton}
    \country{United Kingdom}
    }

    \author{Fidan Mehmeti}
    \email{fzm82@psu.edu}
    \affiliation{
    \institution{Penn State University}
    \state{Pennsylvania}
    \country{United States}
    }

    \author{Sebastian Stein}
    \email{ss2@soton.ac.uk}
    \affiliation{
    \institution{University of Southampton}
    \city{Southampton}
    \country{United Kingdom}
    }

    \author{Tom La Porta}
    \email{tfl12@psu.edu}
    \affiliation{
        \institution{Penn State University}
        \state{Pennsylvania}
        \country{United States}
    }

    \author{Geeth De Mel}
    \email{geeth.demel@uk.ibm.com}
    \affiliation{
        \institution{IBM UK}
        \country{United Kingdom}
    }

    %% By default, the full list of authors will be used in the page
    %% headers. Often, this list is too long, and will overlap
    %% other information printed in the page headers. This command allows
    %% the author to define a more concise list
    %% of authors' names for this purpose.
    \renewcommand{\shortauthors}{Towers et al.}

    %% The abstract is a short summary of the work to be presented in the
    %% article.
    \begin{abstract}
    Edge Cloud Computing enables computational tasks, such as machine learning models or other AI algoriuthms,
    to be processed at the edge of the network using limited
    computational resources in comparison to larger remote data centres. Because of this, resource allocation and
    management is significantly more important. Existing resource allocation approaches usually assume that tasks
    have inelastic resource requirements (i.e., a fixed amount of bandwidth and computation requirements). However,
    this may result in inefficient resource usage due to unbalanced requirements from tasks that can cause
    bottlenecks. To address this, we propose a novel approach that takes advantage of the elastic nature of resources
    as to time taken for an operation to occur is generally proportional to the allocated resource. This, however,
    makes previous research incompatible with such elasticity. Therefore, we describe this problem formally as an
    optimisation problem and propose a scalable approximation algorithm. To deal with self-interested users, we use a
    technique from multi-agent systems and design to demonstrate a centralised auction mechanism that is incentive
    compatible using the approximation algorithm.
    Moreover, we propose a novel Decentralised Iterative Auction that does not require users to reveal their private
    task value. Using extensive simulations, we show that considering the elasticity of resources leads to a gain in
    utility of around 20\% compared to existing approaches, with the proposed approaches typically achieving 95\% of
    the theoretical optimal.
\end{abstract}

    %% The code below is generated by the tool at http://dl.acm.org/ccs.cfm.
    %% Please copy and paste the code instead of the example below.
\begin{CCSXML}
<ccs2012>
   <concept>
       <concept_id>10003752.10010070.10010099.10010107</concept_id>
       <concept_desc>Theory of computation~Computational pricing and auctions</concept_desc>
       <concept_significance>500</concept_significance>
       </concept>
   <concept>
       <concept_id>10010520.10010521.10010537.10003100</concept_id>
       <concept_desc>Computer systems organization~Cloud computing</concept_desc>
       <concept_significance>500</concept_significance>
       </concept>
   <concept>
       <concept_id>10002950.10003624.10003633.10010918</concept_id>
       <concept_desc>Mathematics of computing~Approximation algorithms</concept_desc>
       <concept_significance>100</concept_significance>
       </concept>
 </ccs2012>
\end{CCSXML}

\ccsdesc[500]{Theory of computation~Computational pricing and auctions}
\ccsdesc[500]{Computer systems organization~Cloud computing}
\ccsdesc[100]{Mathematics of computing~Approximation algorithms}
    %% Keywords. The author(s) should pick words that accurately describe
    %% the work being presented. Separate the keywords with commas.
    \keywords{Edge Clouds; Resource Allocation, Resource Elastic Task; Auctions}

    %% This command processes the author and affiliation and title
    %% information and builds the first part of the formatted document.
    \maketitle

    \section{Introduction}\label{sec:introduction}
In the last few years, cloud computing~\cite{cloud_cite} has become a popular solution for running data-intensive
applications remotely. However, large scale data-centres are not a feasible in application domains that require low
latency or high security and privacy. To deal with such domains, \emph{fog/edge computing} has emerged as a
complementary paradigm allowing tasks to be executed at the edge of networks, close to the user, in small data-centers,
known as \emph{edge clouds}.

As the Internet-of-things grows, fog/edge cloud computing is a key enabling technology in particular for applications
in smart cities, disaster response scenarios, home automation systems, etc. In these applications, low-powered devices
generate data or tasks that can't be processed locally but are impractical to use with standard cloud computing
services. More specifically, in smart cities, these devices could be smart traffic light systems that collect data from
roadside sensors to optimise the traffic light sequence to minimise vehicle waiting times; or to analyse videos feeds
from CCTV cameras of suspicious behaviour. In disaster response, sensor data from autonomous vehicles can be aggregated
to produce real-time maps of devastated areas to help first responders better understand the situation and search for
survivors.

To accomplish these tasks, there are typically several types of resources that are needed, including communication
bandwidth, computational power and data storage resources~\cite{vaji_infocom}, and tasks are generally
delay-sensitive, i.e., have a specific completion deadline. When accomplished, different tasks carry different values
for their owners (e.g., the users of IoT devices or other stakeholders such as the police or traffic authority). This
value will depend on the importance of the task, e.g., analysing current levels of air pollution may be less important
than preventing a large-scale traffic jam at peak times or tracking a criminal on the run. Given that edge clouds are
often highly constrained in their resources~\cite{edge_limitations}, we are interested in allocating tasks to edge
cloud servers to maximize the overall social welfare achieved (i.e., the sum of completed task values). This is
particularly challenging, because users in edge clouds are typically self-interested and may behave
strategically~\cite{Bi2019} or may prefer not to reveal private information about their values to a central allocation
mechanism~\cite{Pai2013}.

An important shortcoming of existing work of resource allocation in edge cloud computing is that it assumes tasks have
strict resource requirements -- that is, each task must be allocate a fixed amount of CPU cycles or bandwidth by a
server. This can cause bottlenecks for certain resources when multiple tasks over-request resources. However, in
practice tasks have flexibility in how resources are allocated given the task is computed within a time limit. This
ability to flexibility allocate resource is additionally important in the case of edge computing due to the limited
resource capacities that servers have in comparison to large data-centre. Using this idea of task resource flexibility,
in this paper, we proposed a new optimisation problem that is incompatible with previous research. Therefore three
mechanisms are proposed for the problem: a greedy mechanism, an incentive compatible centralised auction and a novel
decentralised iterative auction that achieve ~95\% of the theoretical optimal.

%% TODO Make an addition to why this flexibility breaks everything previously
    \section{Related work}
\label{sec:related-work}
There is a considerable amount of research in the area of resource allocation and pricing in cloud computing, some of
which use auction mechanisms to deal with competition~\cite{KUMAR2017234,Zhang2017,Du2019,Bi2019}.
However, these approaches assume that users request a fixed amount of resources system resources and processing rates,
with the cloud provider having no control over the speeds, only the servers that the task was allocated to. In our
work, tasks' owners report deadlines and overall data and computation requirements, allowing the edge cloud server to
distribute its resources more efficiently based on each task's requirements.

Other closely related work on resource allocation in edge clouds~\cite{vaji_infocom} considers both the placement of
code/data needed to run a specific task, as well as the scheduling of tasks to different edge clouds. The goal there is
to maximize the expected rate of successfully accomplished tasks over time. Our work is different both in the setup and
the objective function. Our objective is to maximize the value over all tasks. In terms of the setup, they assume that
data/code can be shared and they do not consider the elasticity of resources.

Our problem is related to multidimensional knapsack problems. In particular~\cite{Nip2017}, consider flexibility in
the allocation, with linear constraints that are used for elastic weights. The paper provides a pseudo-polynomial time
complexity algorithm for solving this problem to maximize the values in the knapsack. Our optimisation problem case is
similar to their, but differ due to constraints with our using non-linear not linear constraints.

A majority of approaches taken for task pricing and resource allocation in Cloud Computing uses a fixed resource
allocation mechanism, such that each user requests a fixed amount of resources for a task from a server. However this
mechanism, as previously explained, provides no control for the server over the quantity of resource allocated to a task,
only determining the task's price. As a result, a majority of approaches don't consider the server's management of
resource allocation. Thus research has focused on designing efficient and incentive compatible auction mechanisms.

Work by~\cite{KUMAR2017234} provides a systematic study of double auction mechanisms that are suitable for a range
of distributed systems like Grid computing, Cloud computing, Inter-Cloud systems. The work reviewed 21 different
proposed auction mechanisms over a range of important auction properties like Economic Efficiency,
Incentive Compatibility and Budget-Balance. In a majority of the proposed auction mechanisms, truthfulness was only
considered for the user, thus a Truthful Multi-Unit Double auction mechanism was presented as such that both users and
server should act truthfully.

Some approaches have been taken to increase flexibility within Fog Cloud Computing \citep{Bi2019} through efficient
distribution of data centers and connections to maximise social welfare. A truthful online mechanism was
proposed that was incentive compatible and individually rational, to allow tasks to arrive over time by solving an
integer programming optimisation problem. Similar research in~\cite{vaji_infocom}, considers the placement of code/data
needed to run specific tasks over time where the demands change over time while also considering the operational costs
and system stability. An approximation algorithm achieved 90\% of the optimal social welfare by converting the problem
to a set function optimisation problem.
    \section{Problem formulation}
\label{sec:problem-formulation}
In this section, an outline of the system model describes servers and tasks attributes
(subsection~\ref{subsec:system-model}) used in the resource elastic optimisation problem
(subsection~\ref{subsec:optimisation-problem}). Using this model, we prove that it is NP-Hard
(subsection~\ref{subsec:time-complexity}). Finally, using a example program, we demonstrate in the effectiveness of our
flexible resource allocation scheme compared to an fixed resource allocation scheme used in previous work, as explained
the previous section (subsection~\ref{subsec:example-problem-case}).

\subsection{System model}\label{subsec:system-model}
\begin{wrapfigure}{r}{0.4\linewidth}
    \centering
    \includegraphics[width=\linewidth]{figs/system_model.pdf}
    \caption{System Model}
    \label{fig:system-model}
\end{wrapfigure}
A sketch of the system is shown in Fig.~\ref{fig:system-model}.
We assume that there are a set of servers $I = \{1,2,\ldots,\left|I\right|\}$, which could be accessed either through
cellular base stations or WiFi access points (APs). These servers have different types of limited resources:
storage for the code/data needed to run a task (e.g., measured in Gb), computation capacity in terms of CPU cycles per
time interval (e.g., measured in Ghz), and communication bandwidth to receive the data and to send back the results
of the task after execution per time interval (e.g., measured in Mbit). The servers are assumed to only consider these
attributes for resource allocation, however future work would wish to explore additional attributes like I/O memory
access per second, GPU usage, and more. The servers are also assumed to be heterogeneous in all their characteristics.
Formally, we denote the storage capacity of server $i$ with $S_i$, the computation capacity with $W_i$, and the
bandwidth capacity with $R_i$.

The system also contains a set $J = \{1,2,\ldots,\left| J \right|\}$, of different tasks that each require service from
one of the servers $I$. Each task has a monetary value, denoted $v_j$, representing the maximum price the owner is
willing to pay for the task to be computed. \\
In order to run a task, the server is required to load the appropriate code/data onto the server from a source, then to
compute the code of the task and to send back results to the user. Therefore, for each of these stages, we consider
separate speeds that the operations could occur at to enable the greatest flexibility for the server at each stage. \\
The storage size for the task $j$ is denoted as $s_j$ with the rate that the program is transferred to the server
as $s^{'}_j$. For a task to be computed successfully, it must fetch and execute instructions on a CPU. We consider the
total number of CPU cycles required for the program to be $w_j$, where the number of CPU cycles assigned to the task
per unit of time as $w^{'}_j$. Finally, after the task is run and the results obtained, the latter need to be sent back
to the user. The size of the results for task $j$ is denoted with $r_j$, and the bandwidth used to sent back results to
the user as $r^{'}_j$.

In order to force the server to complete the task with a reasonable time, each task sets a deadline, denoted by $d_j$,
representing the maximum amount of time for a task to be completed successfully within. This includes: the time
required to load the data/code onto the server, run it, and send back results to the user. We assume that there
is an \emph{all}-or-\emph{nothing} task execution reward scheme, meaning that the task value is awarded only if the
task is completed within its deadline.

\subsection{Optimisation problem}
\label{subsec:optimisation-problem}
Given the aforementioned assumptions and variables from the system model, an optimisation problem is constructed as
followed. This problem uses the additional variable $x_{i,j}$ to denote the allocation of a task $j$ that will run on
server $i$.

\begin{align}
    \max & \sum_{\forall j \in J} v_j \left(\sum_{\forall i \in I} x_{i,j}\right) \label{eq:objective} \\
    \mbox{s.t.} \nonumber \\
    & \sum_{\forall j \in J} s_j x_{i,j} \leq S_i, &~ \forall{i \in I} \label{eq:server-storage-constraint} \\
    & \sum_{\forall j \in J} w^{'}_j x_{i,j} \leq W_i, &~ \forall{i \in I} \label{eq:server-computation-constraint} \\
    & \sum_{\forall j \in J} (r^{'}_j + s^{'}_j) \cdot x_{i,j} \leq R_i, &~ \forall{i \in I} \label{eq:server-bandwidth-constraint} \\
    & \frac{s_j}{s^{'}_j} + \frac{w_j}{w^{'}_j} + \frac{r_j}{r^{'}_j} \leq d_j, &~ \forall{j \in J} \label{eq:task-deadline} \\
    & 0 < s^{'}_j, &~ \forall{j \in J} \label{eq:loading-speeds} \\
    & 0 < w^{'}_j, &~ \forall{j \in J} \label{eq:compute-speeds} \\
    & 0 < r^{'}_j, &~ \forall{j \in J} \label{eq:sending-speeds} \\
    & \sum_{\forall i \in I} x_{i,j} \leq 1, &~ \forall{j \in J} \label{eq:server-task-allocation} \\
    & x_{i,j} \in \{0, 1\}, &~ \forall{i \in I},\forall{j \in J} \label{eq:task-allocation}
\end{align}

The objective (eq.~\ref{eq:objective}) is to maximize the total value over all tasks (i.e.,\ social welfare) that
are completed within their deadline (eq.~\ref{eq:task-deadline}).
Constraints~\ref{eq:server-storage-constraint},~\ref{eq:server-computation-constraint}
and~\ref{eq:server-bandwidth-constraint}, prevent over allocation of server resources to allocated tasks.
For the server's storage capacity (constraint~\ref{eq:server-storage-constraint}), each server's storage must be less
than the allocated task's storage requirements. While for the server's computational capacity
(constraint~\ref{eq:server-computation-constraint}), the capacity is limited by a server's allocated tasks
compute resources ($w^{'}_j$). The server's bandwidth capacity (constraint~\ref{eq:server-bandwidth-constraint})
comprises of two parts: the first for loading of data/code of a task onto the server and the second for sending
back result to the user. \\
To force the task to the completed within its assigned deadline, constraint~\ref{eq:task-deadline} required the sum
of time taken for each stages of the task, completed in series, to be less than the deadline value.
Note that if a task is not allocated to any server, this constraints can be satisfied by choosing arbitrarily
resource speed as these resources do not use up any servers' resources in
constraint~\ref{eq:server-storage-constraint},~\ref{eq:server-computation-constraint}
or~\ref{eq:server-bandwidth-constraint}. \\
Constraints~\ref{eq:loading-speeds},~\ref{eq:compute-speeds},~\ref{eq:sending-speeds} enforce that the resource
speeds for each stage ($s^{'}_j$, $w^{'}_j$, and $r^{'}_j$) are all positive and finite.
Finally, as every task can only be served by at most one server, constraints~\ref{eq:server-task-allocation}
and~\ref{eq:task-allocation} enforce this.

This model focus on a single-shot setting where all tasks arrival at the same time to the system. To use this system
in practice where tasks arrival progressively over time, an allocation mechanism would repeat the allocation decisions
described here over regular time intervals. As a result, longer running tasks would reappearing in consecutive time
intervals.
In subsection~\ref{subsec:comparison-between-online-and-batched-resource-allocation}, we evaluate the effectiveness of
such a batching mechanism compared to online mechanisms. We leave a detailed study of online mechanisms to future work.
A probably advantage of such a system is the ability to dynamically change the resource allocation at each time step.

\subsection{Time Complexity}
\label{subsec:time-complexity}
As the optimisation problem as described in Subsection~\ref{subsec:optimisation-problem} is an extension of the
Knapsack problem, a well-studied problem in computer science that known to be NP-Hard. By transforming the problem into
a standard knapsack problem, the time complexity of the problem is also NP-Hard.
\begin{theorem}
    The optimisation problem in subsection~\ref{subsec:optimisation-problem} is NP-hard.
\end{theorem}
\begin{proof}
    The task resource elasticity~\ref{eq:task-deadline} can removed from the optimisation problem to simplify the model
    by setting the task resource speeds to a fixed value that satisfies the deadline constraint. This reduces the model
    to a 0--1 multidimensional knapsack problem~\cite{knapsackproblems_2004}, which is a generalization of a
    simple 0--1 knapsack problem. The latter is an NP-hard problem~\cite{knapsackproblems_2004}. Given this, it follows
    that the 0--1 multidimensional knapsack problem is also NP-hard. Since the optimization problem
    (Eqs.~\ref{eq:objective} -~\ref{eq:task-allocation}) is a generalization of a 0--1 multidimensional knapsack
    problem, it follows that it is NP-hard as well.
\end{proof}

\subsection{Example Problem Case}
\label{subsec:example-problem-case}
\begin{wrapfigure}{l}{0.5\linewidth}
    \centering
    \includegraphics[width=\linewidth]{figs/allocation/optimal_fixed_resource_allocation.png}
    \caption{Optimal solution with fixed resource speeds}
    \label{fig:example-fixed-allocation}
\end{wrapfigure}
Before we propose our allocation mechanisms in the next section, we present an example case to illustrate
why elasticity is important. In this example, there are 12 potential tasks and 3 servers where the flexible solution
is able to achieve 18\% better social welfare compared to the fixed resource allocation solution.
The exact settings can be found in Appendix A with table~\ref{tab:example-tasks-properties}
for the task attributes and table~\ref{tab:example-servers-properties} for the server attributes. \\
The figures~\ref{fig:example-fixed-allocation} and~\ref{fig:example-flexible-allocation} represent each server as a
group of three bars, each relating to each server's resource type, with the percentage of resources used by a task
being the size of the bar.

\begin{figure}
    \centering
    \includegraphics[width=0.5\linewidth]{figs/allocation/optimal_flexible_resource_allocation.png}
    \caption{Optimal solution with elastic resources speeds}
    \label{fig:example-flexible-allocation}
\end{figure}

Figure~\ref{fig:example-fixed-allocation} shows the best possible allocation if tasks have fixed resource speeds (which
were set by minimising the total amount of resources to be completed within the deadline). Here, only 9 of the tasks
are run, resulting in a total social welfare of 980 due to server 1 and 2's limited computational capacity and server
3' limited communication capacity.

In contrast to figure~\ref{fig:example-fixed-allocation}, Figure~\ref{fig:example-flexible-allocation} depicts the
optimal allocation if elastic resources are considered. Here, all the resources are used by the servers, whereas the
fixed example~\ref{fig:example-fixed-allocation} cant do this. In total, the elastic approach manages to schedule all
12 tasks within the resource constraints, achieving a total social welfare of 1200 (an 18\% improvement over the fixed
approach).
    \section{Flexible Resource Allocation Mechanisms}
\label{sec:flexible-resource-allocation-mechanisms}
As explained in the last Section, all previous research outlined in Section~\ref{sec:related-work} is incompatible
with our resource elastic optimisation problem in Section~\ref{sec:problem-formulation}. Therefore in this
section, we propose three mechanisms for solving this optimisation problem: an approximation algorithm and two
auction-based mechanisms.

The optimisation problem is a extended version of the knapsack problem which is often solved using an dynamic
programming method that has pseudo-polynomial time complexity. The solution requires building a table of items to bags
allocation. For our problem, as the resource speeds must be considered at the same time, such a solver can is
infeasible due to both the space and time complexity required.

Because of this issue of allocating both tasks to server and server resources to tasks, this work proposes an
approximation algorithm where tasks are allocated to server with resources in series not in parallel. The centralised
greedy algorithm (detailed in Subsection~\ref{subsec:greedy-algorithm}) ranks tasks that are each allocated to a server
with resources with each stage using an separate ranking function. This algorithm has a social welfare lower bound
of $\frac{1}{\left|J\right|}$ however in practice achieves close to the theoretical optimal while running with
polynomial time complexity.

As task users can be self-interested, they may report their task values or requirements strategically. Traditionally,
VCG~\citep{vickrey,Clarke,groves} is used for such system due to its ability to use any optimisation problem to
calculate a task price. However due to the difficulty of calculating the optimal allocation for this problem, VCG is
infeasible to use in this application. Therefore to deal with self-interested user, we propose two auction-based
mechanisms, an incentive compatible auction using the centralised greedy algorithm
(Section~\ref{subsec:critical-value-auction}). While the second is a one of which is a novel decentralized iterative
auction (Section~\ref{subsec:decentralised-iterative-auction}) that do not require users to reveal the task value.

\subsection{Greedy Algorithm}
\label{subsec:greedy-algorithm}
To solve a knapsack problem, a greedy approximation algorithm is often used that we have extended to this problem
specificity in subsection~\ref{subsec:optimisation-problem}. Because of this elastic nature of task resources means
that an additional stage is required to determine these speeds.

\begin{wrapfigure}{r}{0.6\linewidth}
    \centering
    \includegraphics[width=\linewidth]{figs/allocation/greedy_flexible_resource_allocation.png}
    \caption{Example Greedy allocation using the model from table~\ref{tab:example-tasks-properties}
    and~\ref{tab:example-servers-properties}}
    \label{fig:example-greedy-allocation}
\end{wrapfigure}

More specifically, the greedy algorithm has two stages; stage one sorts the list of tasks based on the value
density of each task that is calculate based on task attributes: value, required resources and deadline. The second
stage uses the sorted list of tasks to iterate through applying two heuristics to select the server based on
available server resources and to allocate resources based on the available server resources and the required resources
of the task. \\
Using the example case from subsection~\ref{subsec:example-problem-case}, the greedy algorithm can complete 11/12 of
the tasks achieving XX\% more social welfare than the fixed solution. This is due to the algorithm being unable to % TODO
consider other tasks resource requirement while allocating resource. %% Todo add greedy heuristics functions

\begin{algorithm}
    \caption{Pseudo code of Greedy Algorithm}
    \label{alg:greedy-mechanism}
    \begin{algorithmic}
        \REQUIRE $J$ is the set of tasks and $I$ is the set of servers
        \REQUIRE $S^{'}_i$, $W^{'}_i$ and $R^{'}_i$ is the available resources
            (storage, computation and bandwidth respectively) of server $i$
        \REQUIRE $v(j)$ is the value density function of task $j$
        \REQUIRE $s(j, I)$ is the server selection function of task $j$ and set of servers $I$ returning the best
            server, or $\emptyset$ if the task is not able to be run on any server
        \REQUIRE $r(j, i)$ is the resource allocation function of a task and server returning the
            loading, compute and sending speeds
        \REQUIRE $\text{sort}(X, f)$ is a function that returns a sorted list of elements in descending order, based
            on a set of elements $X$ and a function for comparing elements $f$

        \STATE{$J^{'} \leftarrow sort(J, v)$}
        \FORALL{$j \in J^{'}$}
            \STATE{$i \leftarrow s(j, I)$}
            \IF{$i \neq \emptyset$}
                \STATE{$s^{'}_j, w^{'}_j, r^{'}_j \leftarrow \gamma(j, i)$}
                \STATE{$x_{i,j} \leftarrow 1$}
            \ENDIF
        \ENDFOR
    \end{algorithmic}
\end{algorithm}

\subsubsection{Greedy Lower Bound}
\label{subsubsec:greedy-lower-bound}
The lower bound of the algorithm is $\frac{1}{\left|J\right|}$ (where $\left|J\right|$ is the number of tasks) with
the task value as the value density function. This lower bound is no affected by the server selection policy or the
resource allocation policy. \\
However in testing, we found that the task value function is not the best value density heuristic as it does not
consider the effect of deadlines or the required resources of the task. In Section~\ref{sec:empirical-results}, we
considered a wide range of heuristics, showing the results of the best heuristics over a range of settings.

\begin{theorem}
    The lower bound of the greedy mechanism is $\frac{1}{n}$ of the optimal social welfare.
\end{theorem}
\begin{proof}
    Due to a task not considering other task's resource requirements then no matter the server selection or resource
    allocation function, it can't be guaranteed that subsequent tasks can be allocated to any server. As a result,
    the algorithm can be guaranteed to achieve at least $\frac{1}{n}$ of the optimal social welfare through using a
    value density function ($v(j) = j_v$). Using this, first task from the sorted task list will have the maximum task
    value meaning the lower bound of the algorithm is $\frac{1}{n}$ of the optimal social welfare.
\end{proof}

\subsubsection{Greedy Time Complexity}
\label{subsubsec:greedy-time-complexity}
Using the greedy mechanism (algorithm~\ref{alg:greedy-mechanism}), the time complexity is polynomial,
$O(\left|J\right| \left|I\right|)$.
\begin{theorem}
    Time complexity of the greedy mechanism is $O(\left|J\right| \left|I\right|)$, where $\left|J\right|$ is the number
    of tasks and $\left|I\right|$ is the number of servers. Assuming that the value density and resource allocation
    heuristics have constant time complexity and the server selection function is $O(\left|I\right|)$.
    %% TODO to check if the resource allocation heuristic is constant time complexity (KKT) probably wrong
\end{theorem}
\begin{proof}
    The time complexity of the stage 1 of the mechanism is $O(\left|J\right| \log(\left|J\right|))$ due to sorting the
    tasks and stage 2 has complexity $O(\left|J\right| \left|I\right|)$ due to looping over all of the tasks and
    applying the server selection and resource allocation heuristics. Therefore the overall time complexity is
    $O(\left|J\right| \left|I\right| + \left|J\right| \log(\left|J\right|) = O(\left|J\right| \left|I\right|)$.
\end{proof}

\subsection{Critical Value Auction}
\label{subsec:critical-value-auction}
Due to the problem case being non-cooperative, if the greedy mechanism was used to allocate resources such that the
value is the price paid. This would be open to manipulation and misreporting of task attributes like the value,
deadline or resource requirements. Therefore in this section we propose an auction that is strategyproof
(weakly-dominant incentive compatible) so users have no incentive to misreport task attributes.

Single-Parameter domain auctions are extensively studied in mechanism design~\cite{nisan2007algorithmic_228} and are
used where an agent's valuation function can be represented as single value. The task price is calculated by finding
the critical value, the minimum task price required for the task to still allocated to a server. This has
been shown to be a strategyproof~\cite{nisan2007algorithmic_229_230} auction making it a weakly-dominant strategy for
a user to honestly reveal a task's attribute.

The auction is implemented using the greedy mechanism from section~\ref{subsec:greedy-algorithm} by finding an initial
allocation using every task's reported value. Then for each task that is allocated, the task price is equal to the
critical value is found by finding the minimum value of the task such that it is still allocated. \\
To find the minimum value is a two step process of removing the task from the list of tasks then running the greedy
mechanism however after each task is allocated. Then a check is done if the critical task could be allocated to any
server. This is a constant time complexity operation by assuming that the server would allocate all of its available
resources for the deadline constant (eq~\ref{eq:task-deadline}). If the task can't be allocated to any server then the
value density of last task allocated is equalled to the required value density of the critical task (this assumes that
in the sorted list, the critical task would appear ahead thus a minor amount could be add thus to guarantee the
critical task is above). Using the value density and the value density function, through finding the inverse of the
value density function, with regards to the task value allows the task critical value to be calculated.

\subsubsection{Critical Value Auction Time Complexity}
\label{subsubsec:critical-value-auction-time-complexity}
The time complexity of the auction is $O(\left|J\right| \left|J\right| \left|I\right|$, the greedy
mechanism repeated $\left|J\right|$ due to calculating each task's critical value.
\begin{theorem}
    The time complexity of the critical value auction is $O(\left|J\right| \left|J\right| \left|I\right|$.
\end{theorem}
\begin{proof}
    The auction uses the greedy mechanism, who's time complexity is $O(\left|J\right| \left|I\right|)$,
    (subsubsection~\ref{subsubsec:greedy-time-complexity}) to find all task's that are allocated to a server. Using the %% Todo, do I really reference the subsubsection
    list of all task's allocated, the critical value of each task must be found. This is done by repeating the
    greedy algorithm for all task (excluding the critical task) with time complexity,
    $O(\left|J\right| \left|I\right|)$, till the critical task can no longer to allocated (a constant time function).
    Where the task's critical value is calculated, a constant time function. As a result, the time complexity for
    calculating the critical value for an individual task is $O(\left|J\right| \left|I\right|)$. Thus the overall time
    complexity is $O(\left|J\right| \left|J\right| \left|I\right|$ due to the critical value possibly being
    found for every task.
\end{proof}

\subsubsection{Critical Value Auction Strategyproof}
\label{subsubsec:critical-value-auction-strategyproof}
In order that the auction is strategyproof, the value density function must be
monotonic~\cite{nisan2007algorithmic_229_230} so that misreporting of any task attributes will result in the value
density decreasing. Therefore a value density function of the form $\frac{v_j d_j}{\alpha(s_j, w_j, r_j)}$ must be used
such that the resulting auction is strategyproof.
\begin{theorem}
    The value density function $\frac{v_j d_j}{\alpha(s_j, w_j, r_j)}$ is monotonic increasing for task $j$ assuming
    the function $\alpha(s_j, w_j, r_j)$ is monotonic increasing for each variable.
\end{theorem}
\begin{proof}
    In order to misreport the task value and deadline, misreported values must be less than their true value. Therefore
    if the value or deadline are decreased then the value density will likewise decrease. \\
    The opposite is true for a task's required resources (storage, compute and result data), as the misreported value
    must be greater than the true value otherwise the task would not be able to be completed. Therefore as the $\alpha$
    function is will increase as the resource requirements increase, the resulting value density will decrease. \\
    So in any case, the overall value density will decrease if the owner doesn't accurately report a task's attribute,
    resulting in the task paying more.
\end{proof}

\subsection{Decentralised Iterative Auction}
\label{subsec:decentralised-iterative-auction}
In some application of edge cloud computing, keeping the value of a task a secret is important for example in
military-tactical networks. Therefore we propose a novel decentralised iterative auction based on the pricing principle
of the VCG auction~\cite{vickrey,Clarke,groves}. VCG auctions calculates the price of an item by finding the
difference in social welfare if the item exists and doesn't exist. Our proposed novel auction uses the same principle,
except in reverse by finding the difference between the current server revenue and the revenue when the task is
required to be allocated with a price of zero. To cause the overall revenue and servers revenue to increase, a small
value called the Price change variable is added to the task price.

Our auction uses this principle by iteratively letting a task advertise its requirements to all of the servers, who
respond with their price to run the task. This price is equal to the server's current revenue minus the solution to the
problem in section~\ref{subsubsec:decentralised-iterative-problem} plus a small value referred to as the price change
variable. The reason for the price change variable is to increase the revenue of the server (otherwise the total
revenue of the server doesn't increase by accepting the task) and is be chosen by the server. Once all of the servers
have responded, the task can compare the minimum server prices to its private value. If the price is less then the
task will accept the server with the lowest price, otherwise the task must stop advertising as the price for the task
to run on any server is greater than its reserve price preventing the task from ever being allocated again.

\begin{algorithm}[H]
    \caption{Decentralised Iterative Auction}
    \label{alg:dia}
    \begin{algorithmic}
        \REQUIRE $I$ is the set of servers
        \REQUIRE $J$ is the set of unallocated tasks, which initial is the set of all tasks to be allocated
        \REQUIRE $P(i, k)$ is solution to the problem in section~\ref{subsubsec:decentralised-iterative-problem}
        using the server $i$ and new task $k$.
        The server's current tasks is known to itself and its current revenue from tasks so not passed as arguments.
        \REQUIRE $R(i, k)$ is a function returning the list of tasks not able to run if task $k$ is allocated to server $i$
        \REQUIRE $\leftarrow_R$ will randomly select an element from a set
        \WHILE{$|J| > 0$}
            \STATE{$j \leftarrow_R J$}
            \STATE{$p, i \leftarrow argmin_{i \in I} P(i, j)$}
            \IF{$p \leq v_j$}
                \STATE{$p_j \leftarrow p$}
                \STATE{$x_{i, j} \leftarrow 1$}
                \FORALL{$j^{'} \in R(i, j)$}
                    \STATE{$x_{i, j^{'}} \leftarrow 0$}
                    \STATE{$p_j^{'} \leftarrow 0$}
                    \STATE{$J \leftarrow J \cup j^{'}$}
                \ENDFOR
            \ENDIF
            \STATE{$J \leftarrow J \setminus \{j\}$}
        \ENDWHILE
    \end{algorithmic}
\end{algorithm}

The algorithm~\ref{alg:dia} is a centralised version of the auction. It works through iteratively checking a currently
unallocated job to find the price if the job was currently allocated on a server. This is done through first solving
the program in section~\ref{subsubsec:decentralised-iterative-problem} which calculates the new revenue if the task was
forced to be allocated with a price of zero. The task price is equal to the current server revenue minus the new
revenue with the task allocated plus a price change variable in order to increase the revenue of the server. The
minimum price returned by $P(i, k)$ is then compared to the job's maximum reserve price (that would be private in the
equivalent decentralised algorithm) to confirm if the job is willing to pay at that price. If the job is willing then
the job is allocated to the minimum price server and the job price set to the agreed price. However in the process of
allocating a job then the currently allocated jobs on the server could be unallocated so these jobs allocation's and
price's are reset then appended to the set of unallocated jobs.

\subsubsection{Server revenue optimisation problem}
\label{subsubsec:decentralised-iterative-problem}
To find the optimal revenue for a server $m$ given a new task $n^{'}$ and set of currently allocated tasks $N$ has a
similar formulation to the optimisation problem in section~\ref{subsec:optimisation-problem}. Except with an additional
variable for the task price $p_n$ for each task $n$.

\begin{align}
    \max & \sum_{\forall n \in N} p_n x_n\label{eq:dia-objective}\\
    \mbox{s.t.} \nonumber \\
    & \sum_{\forall n \in N} s_n x_n + s_{n^{'}} \leq S_m,\label{eq:dia-server-storage-constraint}\\
    & \sum_{\forall n \in N} w^{'}_n x_n + w_{n^{'}} \leq W_m, \label{eq:dia-server-computation-constraint}\\
    & \sum_{\forall n \in N} (r^{'}_n + s^{'}_n) \cdot x_n + (r^{'}_{n^{'}} + s^{'}_{n^{'}}) \leq R_m, \label{eq:dia-server-communication-constraint}\\
    & \frac{s_n}{s^{'}_n} + \frac{w_n}{w^{'}_n} + \frac{r_n}{r^{'}_n} \leq d_n, &~ \forall n \in N \cup \{n^{'}\} \label{eq:dia-task-deadline}\\
    & 0 < s^{'}_n < \infty, &~ \forall{n \in N \cup \{n^{'}\}} \label{eq:dia-loading-speeds}\\
    & 0 < w^{'}_n < \infty, &~ \forall{n \in N \cup \{n^{'}\}} \label{eq:dia-compute-speeds}\\
    & 0 < r^{'}_n < \infty, &~ \forall{n \in N \cup \{n^{'}\}} \label{eq:dia-sending-speeds}\\
    & x_n \in \{0,1\} &~ \forall{n \in N} \label{eq:dia-job-allocation}
\end{align}

The objective (Eq.~\eqref{eq:dia-objective}) is to maximize the price of all currently allocated tasks (not including
the new task as the price is zero). The server resource capacity constraints
(Eqs.~\eqref{eq:dia-server-storage-constraint},~\eqref{eq:dia-server-computation-constraint}
and~\eqref{eq:dia-server-communication-constraint}) are similar to the constraints in the standard model set out in
section~\ref{subsec:optimisation-problem} except with the assumption that the task $n^{'}$ is running so there is no
need to consider if it is running or not. The deadline and non-negative resource speeds constraints
(\ref{eq:dia-task-deadline},~\ref{eq:dia-loading-speeds},~\ref{eq:dia-compute-speeds} and~\ref{eq:dia-sending-speeds})
are all the same equation as the standard formulation for all of the tasks plus the new task. As this formulation only
considers a single server, the task allocation constraint is not consider.

\subsubsection{Decentralised Iterative Auction properties}
\label{subsubsec:decentralised-iterative-auction-properties}
For our proposed auction, we consider four important properties in auction theory.
\begin{itemize}
    \item Budget balanced - True. Since the auction can run without an auctioneer, the auction can be run in a
        decentralised system resulting in no ''middlemen'' taking some money meaning that all revenue goes straight to
        the servers from the tasks.
    \item Individually Rational - True. As the server need to confirm with the task if it is willing to pay an amount
        to be allocated, the task can check this against its secret reserved price preventing the task from ever paying
        more than it is willing.
    \item Incentive Compatible - False. While a task's cannot determine the choices of other task for which server they
        will choose, the order task pricing as this is random or lie about the task value as this information isn't
        revealed. Task's can misreport it's attribute to force other task's to make decisions that would otherwise
        result in the misreporting task from being deallocated from a server. For example, if task misreports some
        attributes it may result in it being cheaper for another task to select a different server. This would mean
        that misreported task from not being deallocated. However for large scale systems, intentionally misreporting
        such attribute is extremely difficult to profit from. This is empirically shown in
        subsection~\ref{subsec:possibility-of-task-mutation-in-decentralised-iterative-auction}.
    \item Economic efficiency - False. The allocation of task's to server is completely random till server becomes full,
        because of this initially allocation and the random selection of task's meaning that often task's result in a
        local maxima rather than the global maxima. As a result, the auction is not 100\% economically efficient
        however the local optima is often close to the global maxima as shown in
        subsection~\ref{subsec:evaluation-of-the-auction-mechanisms}.
\end{itemize}

While the auction is not incentive compatible and that task's do not pay the critical value unlike the critical value
auction, task's do pay the minimal amount for the task to be allocated. This is different from the critical value due
to the requirement that the value is found through a deterministic process, however as this auction randomly selects
task's from the set of unallocated tasks to find a task it can't be the task's critical value.

\subsection{Attributes of the proposed algorithms}
\label{subsec:attributes-of-proposed-algorithms}
In this paper, we have presented three mechanisms to solve the optimisation problem proposed in
section~\ref{subsec:optimisation-problem}. Table~\ref{tab:attributes_algorithms} considers a range of
important attributes of the proposed algorithm to allow easy comparison between the Greedy mechanism,
Critical Value auction and Decentralised Iterative auction.

\begin{table}[H]
    \begin{tabular}{|l|c|c|c|}
        \hline
        \textbf{Attribute} & Greedy Algorithm & Critical Value Auction & Decentralised Iterative Auction \\ \hline
        Truthfulness & & Yes & No \\ \hline
        Optimality & No  & No & No \\ \hline
        Scalability & Yes & Yes & No \\ \hline
        Task information requirements\newline from users & All & All & All except the task value \\ \hline
        Communication over heads & Low & Low & High \\ \hline
        Decentralisation & No  & No  & Yes \\ \hline
    \end{tabular}
    \caption{Attributes of the proposed algorithms: Greedy mechanism, Critical Value auction and
    Decentralised Iterative auction}
    \label{tab:attributes_algorithms}
\end{table}

    \section{Empirical Results}
\label{sec:empirical-results}
To evaluate the algorithms presented in Section~\ref{sec:flexible-resource-allocation-mechanisms},
this sections analysis the different algorithms, the possibility of misreporting task attributes, 
the effect of the server resource capacity and online task arrival. 

To do evaluate our algorithms, synthetic models have been used to generate servers and tasks where
each attribute was taken from a Gaussian distribution. The reason this was done was due to there not
being a De-Facto standard to test cloud computing resource allocation algorithms and that those used
in related work do not consider a deadline for a task.

\subsection{Evaluation of the Greedy Algorithm}
\label{subsec:evaluation-of-the-greedy-algorithm}
To compare the greedy algorithm to the optimal elastic allocation, a branch and bound was implemented to solve the
optimisation problem in section~\ref{subsec:optimisation-problem}. In order to compare to fixed speed equivalent models,
the minimum total resource required to run the job is found and set as the resource speeds for all of the tasks, with
the optimal solution for running the job with the fixed speeds is found as well. To implement the greedy mechanism, the
value density function was $\frac{v_j}{s_j + w_j + r_j}$, server selection was
$\text{argmin}_{\forall i \in I} S^{'}_i + W^{'}_i + R^{'}_i$ and the resource allocation was
$\min s^{'}_j + w^{'}_j + r^{'}_j$ for job $j$ and servers $I$.

% Greedy mechanisms
\begin{figure}[h]
    \centering
    \includegraphics[width=\linewidth]{figs/greedy/greedy_algorithms.png}
    \caption{Comparison of the social welfare for the greedy mechanism, optimal, relaxed problem, time limited branch and bound}
    \label{fig:greedy-mechanism-comparison}
\end{figure}
As figure~\ref{fig:greedy-mechanism-comparison} shows, the greedy mechanism achieves 98\% of the optimal solution for
the small models, the mechanism achieves within 95\% for larger models. In comparison, the fixed allocation achieves
80\% of the optimal solution and always does worse than the social welfare of the greedy mechanism.

\subsection{Evaluation of the Auction mechanisms}\label{subsec:evaluation-of-the-auction-mechanisms}
Figure~\ref{fig:auction-mechanisms-comparison} compares the social welfare of the auction mechanisms: VCG auction,
fixed resource speed VCG auction, critical value auction and the decentralised iterative auction with different price
change variables.

\begin{figure}[h]
    \centering
    \includegraphics[width=\linewidth]{figs/auctions/auctions_results.png}
    \caption{Comparison of the social welfare for the auction mechanisms}
    \label{fig:auction-mechanisms-comparison}
\end{figure}

\subsection{Effectiveness of Decentralised Iterative Auction Heuristics}
\label{subsec:effectiveness-of-decentralised-iterative-auction-heuristics}
\begin{figure}[h]
    \centering
    \includegraphics[width=0.45\linewidth]{figs/dia_heuristics/rounds_grid.png}
    \includegraphics[width=0.45\linewidth]{figs/dia_heuristics/social_welfare_grid.png}
    \caption{Grid search of difference server price change and task initial cost}
    \label{fig:dia_sw_rev_grid_search}
\end{figure}

Within the context of edge cloud computing, the number of rounds for the decentralised iterative auction is important
to making it a feasible auction as it is proportional to the time required to run. We investigated the effect of two
heuristic on the number of rounds and social welfare of the auction; the price change variable and initial cost
heuristic. With an auction using as minimum heuristic values for the price change and initial cost,
figure~\ref{fig:dia_rounds_grid_search}, on average 400 rounds were required for the price to converge while an auction
using a price change of 10 and initial cost of 20 means that only on average 80 rounds are required, 5x less. But by
using high initial cost and price change heuristics, this can prevent tasks from being allocated,
figure~\ref{fig:dia_sw_rev_grid_search}, shows that the difference in social welfare is only 2\% from minimum to
maximum heuristics.

\begin{figure}[h]
    \centering
    \includegraphics[width=0.3\linewidth]{figs/dia_heuristics/revenue_grid.png}
    \caption{Grid search for rounds}
    \label{fig:dia_rounds_grid_search}
\end{figure}

\subsection{Possibility of Task Mutation in Decentralised Iterative Auction}
\label{subsec:possibility-of-task-mutation-in-decentralised-iterative-auction}
%Mutation auction
%\begin{figure}
%    \centering
%    \includegraphics[width=\linewidth]{figs/mutation/}  %% TODO update with the mutated png
%    \caption{Difference in task price to a truthful task and misreported task}
%    \label{fig:auction-mutation}
%\end{figure}
%As the decentralised iterative auction presented in section~\ref{subsec:decentralised-iterative-auction} is not
%incentive compatible, it is possible that misreporting of task attribute can decrease the price paid by a task.
%Figure~\ref{fig:auction-mutation} is a scatter graph of the a task (in orange) and the price of a misreported task
%(in blue). Misreported task were generated by increase the task resource requirements or by decreasing the task value
%or deadline that would substituted for the original truthful task. As the figure clearly shows, in almost all cases of
%mutation causes the

\subsection{Effect of Server Resource Capacity Ratio}
\label{subsec:affect-of-server-resource-capacity-ratio}
Due to the elasticity in the resources, an advantage of such a system is the ability for server to more efficiently
distribute their resources particularly when certain server resources are scarce. To confirm this, a models were
generated where for a range of ratios, a server's bandwidth and computation resources are redistributed to fit the
ratio. At such a point, the greedy algorithm using the settings from
subsection~\ref{subsec:evaluation-of-the-greedy-algorithm} were used, the optimal flexible solution and the optimal
fixed solution.

\begin{figure}[h]
    \centering
    \includegraphics[width=0.45\linewidth]{figs/resource_ratio/social_welfare_percentage.png}
    \includegraphics[width=0.45\linewidth]{figs/resource_ratio/social_welfare_difference.png}
    \caption{Social welfare}
    \label{fig:resource-ratio-social-welfare}
\end{figure}

This can be seen more clear by plotting the average server resource usage each of the ratios.
\begin{figure}[h]
    \centering
    \includegraphics[width=\linewidth]{figs/resource_ratio/server_resource_usage.png}
    \caption{Server resource usage}
    \label{fig:resource-ratio-server-resource-usage}
\end{figure}

\subsection{Batching verse Online task allocation}
\label{subsec:batching-verse-online-task-allocation}
* Explain the reason for the evaluation
* Add figure
* Explanation for DIA advantage as runs over batch

    \section{Conclusion and Future work}\label{sec:conclusion-and-future-work}
In this paper, we studied a resource allocation problem in edge clouds, where resources are elastic and can be
allocated to tasks at varying speeds to satisfy heterogeneous requirements and deadlines. To solve the problem,
we proposed a centralized greedy mechanism with a guaranteed performance bound, and a number of auction-based
mechanisms that also consider the elasticity of resources and limit the potential for strategic manipulation. We show
that explicitly taking advantage of resource elasticity leads to significantly better performance than current
approaches that assume fixed resources.

In future work, while this research considers online based mechanisms
(Subsection~\ref{subsec:comparison-between-online-and-batched-resource-allocation}), this was not the primary
environment these mechanisms were intended to occur within. Therefore additional research will focus primarily on this
online scenario.


    %% The next two lines define the bibliography style to be used, and
    %% the bibliography file.
    \bibliographystyle{ACM-Reference-Format}
    \bibliography{sections/references}

    %% If your work has an appendix, this is the place to put it.
    \appendix
    \section*{Appendix}

\subsection*{Example problem case task and server attributes}
\begin{table}[h]
    \begin{tabular}{|c|c|c|c|}
        \hline
        Name     & $S_i$ & $W_i$ & $R_i$ \\ [0.5ex] \hline
        Server 1 & 400   & 100   & 220   \\ \hline
        Server 2 & 450   & 100   & 210   \\ \hline
        Server 3 & 375   & 90    & 250   \\ \hline
    \end{tabular}
    \caption{Table of server attributes}
    \label{tab:example-servers-properties}
\end{table}

\begin{table}[h]
    \begin{tabular}{|c|c|c|c|c|c|c|c|c|}
        \hline
        Name    & $v_j$ & $s_j$ & $w_j$ & $r_j$ & $d_j$ & $s^{'}_j$ & $w^{'}_j$ & $r^{'}_j$ \\ [0.5ex] \hline
        Task 1  & 100   & 100   & 100   & 50    & 10    & 30        & 27        & 17        \\ \hline
        Task 2  & 90    & 75    & 125   & 40    & 10    & 22        & 32        & 15        \\ \hline
        Task 3  & 110   & 125   & 110   & 45    & 10    & 34        & 30        & 17        \\ \hline
        Task 4  & 75    & 100   & 75    & 35    & 10    & 27        & 21        & 13        \\ \hline
        Task 5  & 125   & 85    & 90    & 55    & 10    & 24        & 28        & 17        \\ \hline
        Task 6  & 100   & 75    & 120   & 40    & 10    & 20        & 32        & 16        \\ \hline
        Task 7  & 80    & 125   & 100   & 50    & 10    & 31        & 30        & 19        \\ \hline
        Task 8  & 110   & 115   & 75    & 55    & 10    & 30        & 22        & 20        \\ \hline
        Task 9  & 120   & 100   & 110   & 60    & 10    & 27        & 29        & 24        \\ \hline
        Task 10 & 90    & 90    & 120   & 40    & 10    & 25        & 30        & 17        \\ \hline
        Task 11 & 100   & 110   & 90    & 45    & 10    & 30        & 26        & 16        \\ \hline
        Task 12 & 100   & 100   & 80    & 55    & 10    & 24        & 24        & 22        \\ \hline
    \end{tabular}
    \caption{Table of task attributes. The columns ($s^{'}_j$, $w^{'}_j$, $r^{'}_j$) are fixed speeds which are not
    considered by the flexible resource allocation.}
    \label{tab:example-tasks-properties}
\end{table}

\subsection*{Online Fixed Resource Allocation Optimisation Problem}

\begin{align}
    \max & \sum_{i \in I, j \in J^{'}} x_{i,j} \label{eq:fixed-env-objective} \\
    \mbox{s.t.} \nonumber \\
    & \sum_{\{j \in J^{'} | a_j \leq t \leq d_j\}} \text{min}(s_j \text{, } s^{'}_j \cdot (t + 1 - a_j)) \cdot x_{i,j} \leq S_i && \forall{i \in I} \label{eq:fixed-server-storage-capacity} \\
    & \sum_{\{j \in J^{'} | a_j \leq t \leq d_j\}} w^{'}_j x_{i,j} \leq W_i && \forall{i \in I} \label{eq:fixed-server-computation-capacity} \\
    & \sum_{\{j \in J^{'} | a_j \leq t \leq d_j\}} (s^{'}_j + r^{'}_j) \cdot x_{i,j} \leq R_i && \forall{i \in I} \label{eq:fixed-server-bandwidth-capacity} \\
    & \sum_{i \in I} x_{i,j} \leq 1 && \forall{j \in J^{'}} \label{eq:fixed-env-allocation-limit} \\
    & x_{i,j} \in \{0, 1\} && \forall{i \in I, j \in J^{'}} \label{eq:fixed-env-allocation-set}
\end{align}
\end{document}
\endinput
%%
%% End of file `sample-manuscript.tex'.

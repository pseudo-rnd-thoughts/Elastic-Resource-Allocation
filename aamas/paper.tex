\section{Introduction}


\section{Related Work}


\section{Problem case}
With the addition of a job deadline and flexibility for the resources allocation for a job by the servers means that
each job has a allocated speed for each resource. Each job ($j$) has it's fixed required resources: storage ($s_j$),
computation ($w_j$) and results data ($r_j$) and the flexibility resource speeds: loading speed ($s^'_j$),
compute speed ($w^'_j$) and sending speed ($r^'_j$). While each server ($i$) has limit available resources like storage
($S_i$), computation ($W_i$) and bandwidth ($R_i$) with each job have a variable of allocation ($x_{ij} \in \{0, 1\}$).
We are considering jobs (J) and servers (I) where we wish maximise the sum of allocated jobs value's making this a
variant of the knapsacking problem.

Optimisation
\begin{equation}
    max \sum^J_j U_j (\sum^I_i x_{ij})
\end{equation}

Jobs servers allocations
\begin{equation}
    \sum^I_i x_{ij} \leq 1 && \forall j = 1,\dots,J
\end{equation}
\begin{equation}
    x_{ij} \in \{0, 1\} && \forall j = 1,\dots,J; i = 1,\dots,I
\end{equation}

Server resource constraints
\begin{equation}
    \sum^J_j S_j x_{ij} \leq S_i && \forall i = 1,\dots,I
\end{equation}

\begin{equation}
    \sum^J_j w_j x_{ij} \leq W_i && \forall i = 1,\dots,I
\end{equation}

\begin{equation}
    \sum^J_j (r_j + s_j) x_{ij} \leq R_i && \forall i = 1,\dots,I
\end{equation}

Deadline
\begin{equation}
    \frac{S_j}{s_j} + \frac{W_j}{w_j} + \frac{R_j}{r_j} \leq D_j && \forall i = 1,\dots,I; j = 1,\dots,J
\end{equation}


\section{Greedy Algorithm}
In previous research, 0-1 multi-dimensional knapsacking problems are shown to have a no fully polynomial time
approximation however this algorithm achieve over 95\% of the optimal solution.

\subsection{Algorithm}

\subsection{Lower bound proof}
If we take each job and server combinatorial and solve find a valid allocation if it is possible then add to a graph
linking the job to the server with a weight of the job value. We can find the maximum weighted matching of jobs to
server that will give a lower bound of m/n where m is the number of servers and n is the number of jobs. \\


\section{Iterative auction}
In real life, server would wish to be payed for the work that they complete with a majority of previous research
considering this case solved through combinatorial double auction however due to the flexibility we are unable to

\subsection{Algorithm}

\subsection{Properties}


\section{Results}
\begin{figure}
    %% \includegraphics{}
    \caption{}
\end{figure}

\subsection{Greedy}
% Optimal, Relaxed, Greedy, Matrix Greedy
% Different model settings and sizes

\subsection{Auctions}
% VCG, Single, Multiple, mutations, CDA
% Different model settings and sizes


\section{Discussion}

